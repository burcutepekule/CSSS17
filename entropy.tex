\documentclass[12pt]{article}
\usepackage{parskip}
\usepackage[a4paper, total={6.25in, 9in}]{geometry}
\usepackage{amsmath,amsfonts,amsthm,bm}
\usepackage[english]{isodate}
\usepackage[final]{pdfpages}
\usepackage{graphicx}
\usepackage{breakcites}
\usepackage{mathtools}
\usepackage[displaymath, mathlines]{lineno}
\usepackage[labelfont=bf,skip=0.5cm,justification=justified,singlelinecheck=false]{caption}
\bibliographystyle{apalike}
%%%%%%%%%%%%%%%%%%%%%%%%%%%%%%%%%%%%%%%%%%%%%%%%%%%%%%%%%%%%%%%%%%%%%%%%%%%%%%%%%%%%
%% Stuff for line numbering %%%%%%%%%%%%%%%%%%%%%%%%%%%%%%%%%%%%%%%%%%%%%%%%%%%%%%%%
%%%%%%%%%%%%%%%%%%%%%%%%%%%%%%%%%%%%%%%%%%%%%%%%%%%%%%%%%%%%%%%%%%%%%%%%%%%%%%%%%%%%
\newcommand*\patchAmsMathEnvironmentForLineno[1]{%
  \expandafter\let\csname old#1\expandafter\endcsname\csname #1\endcsname
  \expandafter\let\csname oldend#1\expandafter\endcsname\csname end#1\endcsname
  \renewenvironment{#1}%
     {\linenomath\csname old#1\endcsname}%
     {\csname oldend#1\endcsname\endlinenomath}}% 
\newcommand*\patchBothAmsMathEnvironmentsForLineno[1]{%
  \patchAmsMathEnvironmentForLineno{#1}%
  \patchAmsMathEnvironmentForLineno{#1*}}%
\AtBeginDocument{%
\patchBothAmsMathEnvironmentsForLineno{equation}%
\patchBothAmsMathEnvironmentsForLineno{align}%
\patchBothAmsMathEnvironmentsForLineno{flalign}%
\patchBothAmsMathEnvironmentsForLineno{alignat}%
\patchBothAmsMathEnvironmentsForLineno{gather}%
\patchBothAmsMathEnvironmentsForLineno{multline}%
}
%%%%%%%%%%%%%%%%%%%%%%%%%%%%%%%%%%%%%%%%%%%%%%%%%%%%%%%%%%%%%%%%%%%%%%%%%%%%%%%%%%%
%%%%%%%%%%%%%%%%%%%%%%%%%%%%%%%%%%%%%%%%%%%%%%%%%%%%%%%%%%%%%%%%%%%%%%%%%%%%%%%%%%%

\title{Evolutionary Dynamics on Dispersal Networks}
\author{\small{Burcu Tepekule, Gregory Britten}}
\date{\small{\printdayoff\today}}

\begin{document}
\maketitle 
%\linenumbers

\subsection*{Abstract}
We consider meta-community stability in the context of the degree distribution for meta-community dispersal networks. In particular we characterize meta-community stability for the case of symmetric and heavy-tailed degree distributions for intra-community interactions (May's community matrix) and the coupled dispersal network in the meta-community generalization. 

\subsection*{Questions}

\begin{itemize}
\item \textbf{mutation-selection balance} How does the entropy distribution change as a function of 1) mutation rate, 2) dispersal rate, 3) selection gradient ($\alpha\delta r$), 4) dispersal network structure (degree distribution), 

\item \textbf{Species interactions} Is `few strong many weak` robust to dispersal network structure? Are other interaction networks more robust for particular dispersal networks? Mutualism vs. antagonism; conjugate interaction-dispersal matrix pairs
\item \textbf{Complexity versus stability}
\end{itemize}

\subsection*{Binary sequences}
\begin{align*}
X_{000} &= [0,0,0]\\
X_{001} &= [0,0,1]\\
\vdots \\
X_{111} &= [1,1,1]\\
\end{align*}

With three bits we have $2^n$ sequences.


\subsection*{Governing ordinary differential equation}

The governing ODE for the logistic quasispecies on a dispersal network is as follows 

\[ \frac{dx_{i}^{l}}{dt} = r_i^l x_i^l\left(1 - \Sigma x^l\right) 
		 + \sum_{\substack{j\\j \neq i}}^N q_0^{h(i,j)}x_j^l
		 - \sum_{\substack{j\\j \neq i}}^N q_0^{h(i,j)}x_i^l
		 + \sum_{\substack{k\\k \neq l}}^{P}m_{i}^{k \rightarrow l}x_{i}^{k}		
		 -  \sum_{\substack{k\\l \neq k}}^{P}m_{i}^{l \rightarrow k} x_{i}^{l} \] 

with parameters and state variable symbols defined in Table X. 

\begin{table}[!h]
\centering
   \caption{Parameter and state variance definitions for the governing ODE} 
\begin{tabular}{lll}
 \textbf{Symbol} &  \textbf{Definition}  \\
 $r_i^l$           & intrinsic growth rate  \\
 $x_i^l$          & concentration of sequence $i$ in patch $l$  \\
 $\Sigma x^l $ & total sequence abundance in path $i$  \\
 $q_0$            & mutation probability  per bit\\
 $h(i,j)$           & Hamming distance between sequences $i$ and $j$\\
 $N$               & Number of unique sequences per patch \\
 $m_i^{h\rightarrow l}$ & dispersal rate of species $i$ from patch $k$ to $l$\\
 $P$  & number of patches 
\end{tabular}
\end{table}

We write the equation in matrix-vector notation
\[ \frac{d \mathbf{x}}{dt} = \mathbf{R\Sigma x}  
		 + \mathbf{Qx}
		 + \mathbf{Mx}\] 
		 
where
\[ \mathbf{R} = \begin{bmatrix}
    r_1^1 & 0        & \dots & \dots & \dots & 0  \\
           0 & r_1^2 & \ddots & \ddots & \ddots & 0 \\
   \vdots & \ddots & \ddots & \ddots & \ddots & \vdots \\
   \vdots & \ddots & \ddots &  r_2^1 & \ddots & \vdots  \\
   \vdots 
           0 &        0 &  \dots  & \dots & r_N^P \\ \end{bmatrix} \]

\subsection*{The Jacobian matrix}
For the purpose of notation, we define
\[ \frac{dX_{i}^{l}}{dt} \equiv f(X_{i}^{l}). \]

The general Jacobian matrix for the meta-community dynamics is then
\[ \mathbf{J} = \begin{bmatrix}
    \frac{\partial f(X_1^1)}{\partial X_1^1} & \frac{\partial f(X_1^1)}{\partial X_2^1} & \dots & \frac{\partial f(X_1^1)}{\partial X_N^1}  &  \frac{\partial f(X_1^1)}{\partial X_1^2} & \frac{\partial f(X_1^1)}{\partial X_2^2} & \dots & \frac{\partial f(X_1^1)}{\partial X_N^{P}} \\ 
    \frac{\partial f(X_2^1)}{\partial X_1^1} &  \frac{\partial f(X_2^1)}{\partial X_2^1} & \dots & \dots & \dots & \dots & \dots & \frac{\partial f(X_2^1)}{\partial X_N^P} \\
    \vdots    & \vdots & \ddots  & \ddots & \ddots & \ddots & \ddots & \vdots\\
    \frac{\partial f(X_N^1)}{\partial X_1^1} & \vdots   & \ddots  & \ddots & \ddots & \ddots & \ddots & \vdots \\
    \frac{\partial f(X_1^2)}{\partial X_1^1} & \vdots & \ddots  & \ddots & \ddots & \ddots & \ddots & \vdots \\
    \frac{\partial f(X_2^2)}{\partial X_1^1} & \vdots & \ddots  & \ddots & \ddots & \ddots & \ddots & \vdots\\
    \vdots   & \vdots & \ddots  & \ddots & \ddots & \ddots & \ddots & \frac{\partial f(X_{N-1}^P)}{\partial X_N^P}\\
    \frac{\partial f(X_N^P)}{\partial X_1^1} & \frac{\partial f(X_N^P)}{\partial X_1^2} & \dots & \dots & \dots & \dots & \frac{\partial f(X_N^P)}{\partial X_{N-1}^P} & \frac{\partial f(X_N^P)}{\partial X_N^P} \\    
\end{bmatrix} \]

where species $1:N$ in patch $l$ are stacked as the top rows, followed by species $1:N$ in patch $l+1$, and so on. 

For the case $N = 2, P=2$ the equations are

\begin{align*}
\frac{dX_1^1}{dt} &=  r_1^1 X_1^1 - a_1^1(X_1^1)^2 +  a_{1,2}^1X_1^1X_2^1 + m_1^{2\rightarrow 1} X_1^2 - m_1^{1\rightarrow 2} X_1^1\\
\frac{dX_2^1}{dt} &=  r_2^1 X_2^1 - a_2^1 (X_2^1)^2 + a_{2,1}^1X_2^1X_1^1  + m_2^{2\rightarrow 1} X_2^2 - m_2^{1\rightarrow 2}X_2^1\\
\frac{dX_1^2}{dt} &=  r_1^2 X_1^2 - a_1^2 (X_1^2)^2 + a_{1,2}^2X_1^2X_2^2 + m_1^{1\rightarrow 2}X_1^1 - m_1^{2\rightarrow 1}X_1^2 \\
\frac{dX_2^2}{dt} &=  r_2^2 X_2^2 - a_2^2 (X_2^2)^2  + a_{2,1}^2X_2^2X_1^2 + m_2^{1\rightarrow 2} X_2^1 - m_2^{2\rightarrow 1} X_2^2\\
\end{align*}

So the Jacobian is 

\begin{align*}
\mathbf{J} &= \begin{bmatrix}
    \frac{\partial f(X_1^1)}{\partial X_1^1} & \frac{\partial f(X_1^1)}{\partial X_2^1} & \frac{\partial f(X_1^1)}{\partial X_1^2} & \frac{\partial f(X_1^1)}{\partial X_2^2}\\ 
        \frac{\partial f(X_2^1)}{\partial X_1^1} & \frac{\partial f(X_2^1)}{\partial X_2^1} & \frac{\partial f(X_2^1)}{\partial X_1^2} & \frac{\partial f(X_2^1)}{\partial X_2^2}\\ 
            \frac{\partial f(X_1^2)}{\partial X_1^1} & \frac{\partial f(X_1^2)}{\partial X_2^1} & \frac{\partial f(X_1^2)}{\partial X_1^2} & \frac{\partial f(X_1^2)}{\partial X_2^2}\\ 
                \frac{\partial f(X_2^2)}{\partial X_1^1} & \frac{\partial f(X_2^2)}{\partial X_2^1} & \frac{\partial f(X_2^2)}{\partial X_1^2} & \frac{\partial f(X_2^2)}{\partial X_2^2} \\    
\end{bmatrix} \\
 &= \scriptsize{\begin{bmatrix}
    r_1^1 - 2a_1^1X_1^1 + a_{1,2}^1X_2^1 - m_1^{1\rightarrow 2} &  a_{1,2}^1 X_1^1 & m_1^{2\rightarrow 1} & 0 \\ 
    a_{1,2}^1X_2^1 & r_2^1 - 2a_2^1X_2^1 + a_{2,1}X_1^1 - m_2^{1\rightarrow 2} & 0 & m_2^{2\rightarrow 1} \\
    m_1^{1\rightarrow 2} & 0 & r_1^2 -2a_1^2X_1^2 + a_{1,2}^2X_2^2  - m_1^{2\rightarrow 1}& a_{1,2}^2X_1^2 \\
    0 & m_2^{1\rightarrow 2} & a_{2,1}^2X_2^2 & r_2^2 - 2a_2^2X_2^2 + a_{2,1}^2X_1^2-m_2^{2\rightarrow 1}\\       
\end{bmatrix} }
\end{align*}

We partition the Jacobian into intra-community interaction and meta-community dispersal components
\[ \mathbf{J} = \mathbf{J_A} + \mathbf{J_M} \]

where
\begin{align*}
\mathbf{J_A} &= \small{\begin{bmatrix}
    r_1^1 - 2a_1^1X_1^1+a_{1,2}^1X_2^1 & a_{1,2}^1X_1^1 & 0 & 0 \\ 
    a_{1,2}^1X_2^1 & r_2^1 - 2a_2^1X_2^1 + a_{2,1}X_1^1 & 0 & 0 \\
    0 & 0 & r_1^2 -2a_1^2X_1^2 + a_{1,2}^2X_2^2  & a{1,2}^2X_1^2 \\
    0 & 0 & a_{2,1}^2X_2^2 & r_2^2 - 2a_2^2X_2^2 + a_{2,1}^2X_1^2 \\       
\end{bmatrix}} \\
 \mathbf{J_M} &= \begin{bmatrix}
    -m_1^{1\rightarrow 2} & 0 & m_1^{2\rightarrow 1} & 0 \\ 
    0 & -m_2^{1\rightarrow 2}& 0 & m_2^{2\rightarrow 1} \\
    m_1^{1\rightarrow 2} & 0 & -m_1^{2\rightarrow 1} & 0 \\
    0 & m_2^{1\rightarrow 2} & 0 & -m_2^{2\rightarrow 1} \\       
\end{bmatrix} 
\end{align*}

We can further decompose $\mathbf{J_M}$ into emigration and immigration networks
\[ \mathbf{J_M} = \mathbf{J_{M_e}} + \mathbf{J_{M_i}} \] 

where the emigration matrix is the diagonal matrix $\mathbf{J_{M_e}}$ and the immigration matrix $\mathbf{J_{M_i}}$ forms the off-diagonal terms. 

\subsection*{General community dynamics}
For a general community dynamics model $g$ in detailed balance, 
\[ \tilde{a}_{i,j}^l = \frac{\partial g(X_i^l)}{\partial X_j^l}\Bigr|_{\substack{{X_i^l=X_i^l}^*\\{X_j^l=X_j^l}^*}}  \]

\subsection*{Degree distributions}
Here we consider meta-community stability as a function of the probability distribution for the underlying network components. We are particularly interested in how dispersal network degree distribution models modulate the meta-community stability in the context of different intra-community interactions networks.

We make a distinction between the graphical structure of the interaction and dispersal networks, versus the structure of the resulting Jacobian at detailed balance. 

\subsubsection*{Dispersal networks}
\subsubsection*{Interactions networks}

\bibliography{library}
%%%%%%%%%%%%%%%%%%%%%%%%%%%%%%%%%%%%%%%%%%%%%%%%%%%%%%%%%%%%%%%%%%%%%%%%
%% Figures %%%%%%%%%%%%%%%%%%%%%%%%%%%%%%%%%%%%%%%%%%%%%%%%%%%%%%%%%%%%%
%%%%%%%%%%%%%%%%%%%%%%%%%%%%%%%%%%%%%%%%%%%%%%%%%%%%%%%%%%%%%%%%%%%%%%%%
%\newpage
%\begin{figure}[h]
%    \begin{center}
%       \includegraphics[scale=]{}
%    \end{center}
%\caption[\textbf{}]{\textbf{}. }
%\label{Figure1}
%\end{figure}
\end{document}